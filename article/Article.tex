%
% Complete documentation on the extended LaTeX markup used for Insight
% documentation is available in ``Documenting Insight'', which is part
% of the standard documentation for Insight.  It may be found online
% at:
%
%     http://www.itk.org/

\documentclass{InsightArticle}


%%%%%%%%%%%%%%%%%%%%%%%%%%%%%%%%%%%%%%%%%%%%%%%%%%%%%%%%%%%%%%%%%%
%
%  hyperref should be the last package to be loaded.
%
%%%%%%%%%%%%%%%%%%%%%%%%%%%%%%%%%%%%%%%%%%%%%%%%%%%%%%%%%%%%%%%%%%
\usepackage[dvips,
bookmarks,
bookmarksopen,
backref,
colorlinks,linkcolor={blue},citecolor={blue},urlcolor={blue},
]{hyperref}
% to be able to use options in graphics
\usepackage{graphicx}
% for pseudo code
\usepackage{listings}
% subfigures
\usepackage{subfigure}


%  This is a template for Papers to the Insight Journal. 
%  It is comparable to a technical report format.

% The title should be descriptive enough for people to be able to find
% the relevant document. 
\title{Morphology with parabolic structuring elements}

% Increment the release number whenever significant changes are made.
% The author and/or editor can define 'significant' however they like.
\release{0.00}

% At minimum, give your name and an email address.  You can include a
% snail-mail address if you like.
\author{Richard Beare}
\authoraddress{Richard.Beare@med.monash.edu.au\\Department of Medicine\\Monash University\\Melbourne\\Australia}

\begin{document}
\maketitle

\ifhtml
\chapter*{Front Matter\label{front}}
\fi


\begin{abstract}
\noindent
Morphological erosion and dilation filters employ a structuring
function, with flat structuring functions being the most common
example. Parabolic\footnote{This article will use the term
``parabolic'', but much of the literature uses ``quadratic''}
structuring functions are less well known but theoretically very
important and practically very useful. This paper briefly introduces
morphology using parabolic structuring functions, describes the ITK
classes used to implement them and includes a number of sample
applications.
\end{abstract}

\tableofcontents

\section{Introduction}
Parabolic structuring functions (PSF) have the following important properties:
\begin{itemize}
\item They are closed under dilation - i.e. dilating two PSFs in a third PSF.
\item An n-dimensional PSF can be obtained by combining n
one-dimensional PSFs in independent directions.
\item PSFs are rotationally symmetric allowing the dimensional
decomposition by dilation.
\end{itemize}
These properties make PSFs the morphological counterpart of the
Gaussian in linear image processing \cite{Boomgaard96}.

The dimensional decompostion properties lead to efficient algorithms
for implementing parabolic morphological operations.

Parabolic morphology operations are useful in image sharpening,
distance transforms, are a less vigorous alternative to conventional
morphological operations based on flat structuring elements and a
potentially useful, faster, alternative to shaped structuring elements
such as the ``rolling ball'' often used in background estimation in
tools such as ImageJ.

\section{Brief background theory}
A dilation by a one dimensional parabolic structuring element is
illustrated in Figure \ref{fig:paradilate}. The dilation of the signal
at point A is found by lowering the structuring function centred at A
until the structuring function comes into contact with the signal, in
this case at point B. The dilation at this point is given by point C,
the height of the parabola turning point. The equivalent erosion is
calculated by raising an inverted parabola into the signal from below.

\begin{figure}[htbp]
\centering
\includegraphics[scale=0.75]{dilate_parabolic}
\caption{Dilation of a 1D signal with a parabolic structuring element.\label{fig:paradilate}}
\end{figure}


\appendix



\bibliographystyle{plain}
\bibliography{local,InsightJournal}
\nocite{ITKSoftwareGuide}

\end{document}

